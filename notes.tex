%        File: defense_notes.tex
%     Created: Tue Jul 30 03:00 PM 2013 C
% Last Change: Tue Jul 30 03:00 PM 2013 C
%
\documentclass[letterpaper]{article}
\usepackage[top=1.0in,bottom=1.0in,left=1.0in,right=1.0in]{geometry}
\usepackage{xspace}
\usepackage{verbatim}
\usepackage{amssymb}
\usepackage{graphicx}
\usepackage{longtable}
\usepackage{amsfonts}
\usepackage{amsmath}
\usepackage[usenames]{color}
\usepackage[
naturalnames = true, 
colorlinks = true, 
linkcolor = Black,
anchorcolor = Black,
citecolor = Black,
menucolor = Black,
urlcolor = Blue
]{hyperref}
\def\thesection       {\arabic{section}}
\def\thesubsection     {\thesection.\alph{subsection}}
\newcommand{\Cyder}{\textsc{Cyder}\xspace}
\newcommand{\Cyclus}{\textsc{Cyclus}\xspace}

\author{K. Huff
\\ \href{mailto:khuff@cae.wisc.edu}{\texttt{khuff@cae.wisc.edu}}
}

\date{}
\title{Defense Notes}
\begin{document}
\maketitle

\section*{Title}

Introduce yourself.

\section*{Outline}



\section*{Top Level Fuel Cycle Simulators}

Nuclear fuel cycle simulation seeks to provide a system level overview of the 
movement of materials and other quantities of interest among facilities as 
nuclear energy is produced. Using fuel cycle simulation tools, the potential 
impact of novel technologies and policies can be approximated. 

\Cyclus is one such simulation tool. The \Cyder tool, which is the primary 
subject of this work, represents the geologic repository, here.

\section*{Future Fuel Cycle Options}

A number of potential nuclear fuel cycles are being considered domestically. 

\section*{Disposal Geology Options Considered}

A number of potential nuclear waste repository geologies are being considered 
domestically. 

\section*{Repository Components}



\section*{Need for an Integrated Repository Model}

An integrated repository performance model enables the analysis of global fuel
cycle metrics concerned with material routing, intermediate storage time, and
repository performance. Post-processed analysis of these metrics can neglect
feedback effects of geologic disposal capacity constraints and repository
performance within the fuel cycle system.


\section*{Outline}


\section*{Waste Stream Acceptance}
The Cyder Facility dynamically accepts material from the coupled fuel
cycle simulation. The capacity decision is the interface at which feedbacks 
occur.

\section*{Waste Form Conditioning}

Waste conditioning is the process of packing a waste stream into an appropriate 
waste form.
In Cyder, discrete waste streams are conditioned into the appropriate
discrete waste form according to user-specified pairings.

\section*{Waste Packaging}
Waste packaging is the process of placing one or many waste forms into a 
containment package.
In Cyder, one or more waste forms are loaded into the appropriate waste
package according to user-specified pairings. 


\section*{Wate Package Emplacement}

Finally, the waste package is emplaced in a buffer component, which contains 
many other waste packages, spaced evenly in a grid. The grid is defined by the 
user input and depends on repository depth, z, waste package spacing, x, and 
tunnel spacing, y as in Figure 10.

\section*{Cyder Paradigm: Modularity}
A modular repository framework facilitates
\begin{itemize}
\item interchangable subcomponents
\item and simulations with varying levels of detail.
\end{itemize}
Integration with a fuel cycle simulator facilitates
\begin{itemize}
\item analysis of feedback effects upon the fuel cycle
\item and fuel cycle impacts on disposal system performance.
\end{itemize}

\section*{<++>}
\section*{<++>}
\section*{<++>}
\section*{<++>}
\section*{<++>}
\section*{<++>}
\section*{<++>}
\section*{<++>}
\section*{<++>}

  \section*{Nested Components}
  The NuclideModel in a Component can be interchangeably represented by any of 
  the four nuclide transport models. 

  \section*{Advection Dispersion Equation}
<++>

  \section*{Component Interfaces}
Solutions to this equation can be categorized by their boundary conditions and 
those boundary conditions serve as the interfaces between components in the 
Cyder library of nuclide transport models.

\section*{Implicit Timestepping}
Each Component passes some information radially outward to the nested 
Component immediately containing it and some information radially 
inward to the nested Component it contains. 

Mass distribution in Component 0 at time $t_n$ is found from the inner boundary 
condition at time $t_n$ and the outer boundary condition at $t_{n-1}$. Outer 
boundary conditions are solved for numerically at $t_n$.

\section*{Clay GDSM Model.}
explain

  \section*{Radionuclide Transport: Degradation Rate Based Release}
In this model, the contaminants in the degraded fraction of the control volume 
are available to adjacent components. The available contaminants
$m_{ij}(t)$, at the boundary between cell $i$ to cell $j$ at time $t$ are thus

  \section*{Radionuclide Transport: Degradation Rate Based Release}
<++>

  \section*{Radionuclide Transport : Mixed Cell with Sorption and Solubility}
<++>

  \section*{Radionuclide Transport : Mixed Cell Sorption}
<++>
The mass of contaminant sorbed into the degraded and precipitated solids can be
found using a linear isotherm model \cite{schwartz_fundamentals_2004},
characterized by the relationship 

  \section*{Radionuclide Transport : Mixed Cell Solubility Limitation}
In addition to engineered barriers, contaminant transport is constrained by 
  the solubility limit \cite{hedin_integrated_2002}, 

  \section*{Radionuclide Transport: Lumped Parameter Transport Model}
The method by which each lumped parameter component is modeled is
according to a relationship between the incoming concentration, $C_{in}(t)$,
and the outgoing concentration, $C_{out}(t)$.

  \section*{Radionuclide Transport: Lumped Parameter Transport Model}
Selection of the response function is usually based on experimental tracer
results in the medium at hand. However, some functions used commonly in
chemical engineering applications \cite{maloszewski_lumped_1996} include the
Piston Flow Model (PFM), Exponental Model (EM), and Dispersion Model (DM). 

  \section*{Radionuclide Transport: Lumped Parameter Transport Model}
The solutions to these for constant concentration at the 
source boundary are given by Maolszewski and Zuber.

  \section*{Radionuclide Transport: 1D Finite, Cauchy B.C.}
  A one dimensional, finite, unidirectional flow,
  solution with Cauchy and Neumann boundary conditions

  \section*{Radionuclide Transport: 1D Finite, Cauchy B.C.}
For the boundary conditions, 

  \section*{Radionuclide Transport: 1D Finite, Cauchy B.C.}
For the vertical flow coordinate system, $A$ is defined as

\section*{Demonstration}
<++>
\section*{Degradation Rate Model Base Case I}
<++>
\section*{Degradation Rate Model Base Case I}
<++>
\section*{Degradation Rate Model Base Case II}
<++>
\section*{Degradation Rate Model Base Case II}
<++>
\section*{Degradation Rate Model Base Case III}
<++>
\section*{Degradation Rate Model Base Case III}
<++>
\section*{Degradation Rate Model Base Case IV}
<++>
\section*{Degradation Rate Model Base Case IV}
<++>
\section*{Mixed Cell Model Base Case  II}
<++>
\section*{Mixed Cell Model Base Case  II}
<++>
\section*{Lumped Parameter Model Base Case}
<++>
\section*{1dppm Base Case I}
<++>
\section*{1dppm Base Case I}
<++>
\section*{Clay GDSM Solubility Sensitivity}
For solubility limits above the threshold, increase to the limit had no effect on the peak dose. This demonstrates the 
situation in which the solubility limit is so high that even complete 
dissolution of the waste inventory into the pore water is insufficient to reach 
the solubility limit.
\section*{Cyder Solubility Sensitivity}
In the parametric analysis of \Cyder performance, it was shown that the 
solubility sensitivity behavior closely matched that of the GDSM 
sensitivity behaviors. In agreement with expectations, \Cyder results in Figure 
\ref{fig:sol_result}, demonstrate a sharp turnover 
where the solubility limit exceeds the point at which it limits movement. 

\section*{Clay GDSM Sorption Sensitivity}
It is clear from Figures \ref{fig:KdSumFactor} and \ref{fig:KdSum} that 
for retardation coefficients greater than the threshold , the 
relationship between peak annual dose and retardation coefficient is a strong 
inverse one. 
\section*{Cyder Sorption Sensitivity}

In Figure \ref{fig:kd_result}, increasing the retardation 
coefficient results in a smooth but dramatic turnover. 


\section*{Clay GDSM Degradation Rate Sensitivity}

Highly soluble and non-sorbing $^{129}I$ demonstrates a direct proportionality between dose rate and 
fractional degradation rate until a turnover where other natural system 
parameters dampen transport. Highly soluble and non-sorbing $^{129}I$ demonstrates a direct 
proportionality to the inventory multiplier.

$^{129}I$ waste form degradation rate sensitivity.
$^{129}I$ inventory multiplier sensitivity.


\section*{Cyder Degradation Rate Sensitivity}
In the parametric sensitivity analysis conducted with the \Cyder tool, waste 
form degradation rate sensitvity similarly shows the two regimes noted in the 
GDSM analysis.  

Sensitivity demonstration of the degradation rate in \Cyder for an 
arbitrary isotope.


\section*{GDSM Model Advective Diffusive Sensitivity}
For vertical advective velocities 
$6.31\times10^{-6}[m/yr]$ and above, lower reference diffusivities are 
ineffective at attenuating the mean of the peak doses for soluble, non-sorbing 
elements. 

\section*{Cyder Advective Diffusive Sensitivity}
Increased advection and increased diffusion lead to greater release. Also, when 
both are varied, a boundary between diffusive and advective
regimes can be seen. An example of these results are shown here.
Advection vs. Diffusion Sensitivity in Cyder. Dual advective velocity 
and reference diffusivity sensitivity for a non-sorbing, infinitely soluble 
nuclide.

  \section*{Thermal Base Case Demonstration}
A validation exercise comparing the combined scaling and  
superposition calculations demonstrates an average error of 1.1\% and a 
maximum error of 4.4\%.

This comparison of STC calculated thermal response from $Cm$ 
inventory per MTHM in 51GWd burnup UOX PWR fuel compares favorably with results 
from the semi-analytic model from LLNL.


  \section*{Thermal Base Case Demonstration}
Here, percent error is 
\begin{align}
\mbox{ percent error } &= 100\times\frac{\left|\Delta T_{LLNL} - \Delta 
T_{STC}\right|}{ \Delta T_{LLNL}}.
\end{align}

\section*{LLNL Model Waste Package Spacing Sensitivity}
  Figure \ref{fig:Cm242spacing_sens} shows the trend in which increased waste 
  package spacing of a medium decreases areal thermal energy 
  deposition in the near field.
  $K_{th}$ Sensitivity to $s$.
  
  Increased waste package 
  spacing decreases areal thermal energy deposition 
  (here represented by STC) in the near field (here $r_{calc} = 0.5m$).


\section*{LLNL Model Limiting Radius Sensitivity}
  The location of the limiting radius has a strong effect on the 
  waste package loading limit, for a fixed limiting temperat
  
  $K_{th}$ Sensitivity to $r_{lim}$.
  
  Increased limiting radius decreases thermal energy deposition contributing to 
  the thermal limit (here represented by STC).

\section*{Cyder Spacing and Limiting Radius Sensitivity}
  The thermal diffusivity was compared both with the 
  spacing between waste packages and the limiting radius. 

  $alpha_{th}$ vs. $r_{lim}$ Sensitivity in Cyder.

  Cyder results agree with those of the LLNL model. The importance of the 
  limiting radius decreases with increased $K_{th}$. The above example thermal 
  profile results from 10kg of $^{242}Cm$.

\section*{LLNL Model Thermal Conductivity Sensitivity}
By varying the thermal conductivity of the repository model from 0.1 to 4.5 
$[W\cdot m^{-1} \cdot K^{-1}]$, this sensitivity analysis succeeds in capturing 
the domain of thermal conductivities witnessed in high thermal conductivity 
salt deposits as well as low thermal conductivity clays.
$K_{th}$ Sensitivity for Low $\alpha_{th}$ in LLNL Model.
Increased thermal conductivity decreases thermal energy deposition 
(here represented by STC) in the near field (here $r_{calc} = 0.5m$).

\section*{Cyder Thermal Conductivity and Limiting Radius Sensitivity}

These figures validate the trend noted above that 
increased thermal conductivity of a medium decreases thermal energy deposition 
in the near field. Additionally, analysis with the \Cyder STC database 
demonstrates the way in which the importance of spacing and the importance of 
the limiting radius decrease with increasing $K_{th}$.

$K_{th}$ vs. $r_{lim}$ Sensitivity in Cyder
Cyder results agree with 
those of the LLNL model. The importance of the limiting radius decreases with 
increased $K_{th}$. The above example thermal profile results from 10kg of 
$^{242}Cm$.

\section*{Cyder Thermal Conductivity and Limiting Radius Sensitivity}

$K_{th}$ vs. Waste Package Spacing Sensitivity in Cyder.
Cyder results 
agree with 
those of the LLNL model. The importance of the limiting radius decreases with 
increased $K_{th}$. The above example thermal profile results from 10kg of 
$^{242}Cm$




\section*{LLNL Model Thermal Diffusivity Sensitivity}
  By varying the thermal diffusivity of the disposal system from $0.1-3\times 
  10^{-6} [m^2\cdot s^{-1}]$, this sensitivity analysis succeeds in capturing 
  the domain of 
  thermal diffusivities witnessed in high thermal diffusivity salt deposits as 
  well as low thermal diffusivity clays.
    Increased thermal diffusivity decreases thermal energy deposition (here represented by STC) in the near field (here $r_{calc} = 0.5m$).


\section*{Cyder Thermal Diffusivity and Conductivity Sensitivity}
$\alpha_{th}$ vs. $K_{th}$ Sensitivity in Cyder. Cyder trends agree
  with those of the LLNL model, in which increased thermal diffusivity results 
  in 
  decreased thermal depsoition in the near field. The above example thermal 
  profile results from 10kg of $^{242}Cm$.

\section*{Cyder Thermal Diffusivity and Limiting Radius Sensitivity}
  Further \Cyder analysis shows the importance of $K_{th}$ remains constant, 
  but 
  the importance of the limiting radius decreases with increasing 
  $\alpha_{th}$.
  $alpha_{th}$ vs. $r_{lim}$ Sensitivity in Cyder.
  Cyder trends agree with 
  those of the LLNL model. The importance of the limiting radius decreases with 
  increased $K_{th}$. The above example thermal profile results from 10kg of 
  $^{242}Cm$

\pagebreak
\bibliographystyle{ieeetr}
\bibliography{paper}
\end{document}


