%        File: defense_notes.tex
%     Created: Tue Jul 30 03:00 PM 2013 C
% Last Change: Tue Jul 30 03:00 PM 2013 C
%
\documentclass[letterpaper]{article}
\usepackage[top=1.0in,bottom=1.0in,left=1.0in,right=1.0in]{geometry}
\usepackage{xspace}
\usepackage{verbatim}
\usepackage{amssymb}
\usepackage{graphicx}
\usepackage{longtable}
\usepackage{amsfonts}
\usepackage{amsmath}
\usepackage[usenames]{color}
\usepackage[
naturalnames = true, 
colorlinks = true, 
linkcolor = Black,
anchorcolor = Black,
citecolor = Black,
menucolor = Black,
urlcolor = Blue
]{hyperref}
\def\thesection       {\arabic{section}}
\def\thesubsection     {\thesection.\alph{subsection}}
\newcommand{\Cyder}{\textsc{Cyder}\xspace}
\newcommand{\Cyclus}{\textsc{Cyclus}\xspace}

\author{K. Huff
\\ \href{mailto:khuff@cae.wisc.edu}{\texttt{khuff@cae.wisc.edu}}
}

\date{}
\title{Defense Notes}
\begin{document}
\maketitle

\section{Title}
\begin{itemize}
\item University of Chicago, BA in Physics 2008.
\item Joined the Nuclear Engineering Program at the University of Wisconsin Madison in 2008 
\item Passed my Preliminary Examination just shy of two years ago, September 1, 2011.
\item Have spent those years at Argonne National Laboratory as a Laboratory 
Graduate Fellow with the Used Fuel Disposition Campaign.
\end{itemize}
\section{Outline}

First, we'll address motivation for this work.

\section{Top Level Fuel Cycle Simulators}

Nuclear fuel cycle simulation seeks to provide a system level overview of the 
movement of materials and other quantities of interest among facilities as 
nuclear energy is produced. Using fuel cycle simulation tools, the potential 
impact of novel technologies and policies can be approximated. \Cyclus is one 
such simulation tool. The \Cyder tool, which is the primary subject of this 
work, represents the geologic repository at the back end of the cycle in this 
diagram.

\section{Future Fuel Cycle Options}

A number of potential nuclear fuel cycles are being considered domestically. 
These upstream decisions affect the material destined for the repository and 
present an array of challenges for emplacement and containment.

\section{Disposal Geology Options Considered}

A number of geologic media for potential nuclear waste repositories are being 
considered domestically. Most geologic media being considered internationally are 
reducing, saturated concepts.  

\section{Repository Components}

Repository concepts in these various geologic media accordingly have various 
engineered barrier choices. These are the first containment barriers that are 
breached in a repository simulation.

\section{Need for an Integrated Repository Model}

An integrated repository performance model enables the analysis of global fuel 
cycle metrics concerned with material routing, intermediate storage time, and 
repository performance.  Without integration, only post-processed analysis of 
these metrics is possible. This method can neglect feedback effects of geologic 
disposal capacity constraints and repository performance within the fuel cycle 
system in addition to being more cumbersome to use. However, no fuel cycle 
simulators have integrated repository performance until now. 

\section{Outline}

Now, I will discuss the methodology with which contaminant and thermal 
transport are modeled within the tool I've written.

\section{Waste Stream Acceptance}

The Cyder Facility dynamically accepts material from the coupled fuel
cycle simulation. The capacity decision is the interface at which feedbacks 
occur.

\section{Waste Form Conditioning}

Waste conditioning is the process of packing a waste stream into an appropriate 
waste form.  In Cyder, discrete waste streams are conditioned into the 
appropriate discrete waste form according to pairings that are specified by the 
user within the input file.

\section{Waste Packaging}

Waste packaging is the process of placing one or many waste forms into a 
containment package.  In Cyder, one or more waste forms are loaded into the 
appropriate waste package. Again, the appropriate waste package is selected 
according to user-specified pairings in the input file. 


\section{Wate Package Emplacement}

Finally, the waste package is emplaced in a buffer component, which contains 
many other waste packages. During this emplacement, the waste packages are 
spaced evenly in a grid. The grid is defined by the user input and depends on 
repository depth, z, waste package spacing, x, and tunnel spacing, y as in 
the figure.

\section{Cyder Paradigm: Modularity}

If we look down along the z axis at the footprint of the repository concept 
modeled in cyder, we see that each of these barriers is treated as a distinct 
component in Cyder. This allows Cyder to modularly encapsulate the behaviors of 
the various components I just addressed. In order to facilitate interchangable 
subcomponents and simulations with varying levels of detail, it treats each of 
the barrier components of the repository as distinct simulation objects. 


\section{Cyder Paradigm: Modularity}

If we look individually at each component..

\section{Cyder Paradigm: Modularity}

we see that in \Cyder , each component has its own contaminant transport model 
and its own thermal transport model. These, as well as the data that the user 
provides to parameterize them, determine the physics approximated within each 
component. 

\section{Nested Components}
So, to review, in \Cyder , each Component has : 
  \begin{itemize}
    \item a Geometry to describe its dimensions and location
    \item a NuclideModel for contaminant transport 
    \item a ThermalModel for heat transport
    \item a Parent Component at its external barrier
    \item one or more Daughter Components at its internal barrier
  \end{itemize}

  Components have other data members such as a Type (WF, WP, BUFFER, FF), a 
  material data table, a start date, etc. 

\section{Thermal Modeling in Cyder}
Two types of thermal modeling occur in Cyder. 
\begin{itemize}
\item The first is \textbf{capacity estimation} for waste stream acceptance.
\item The next is \textbf{heat evolution} which determines heat evolution in 
the modules over repository lifetime.
\end{itemize}

\section{Thermal Modeling in Cyder}
Each can be acheived with one thermal model,
\begin{itemize}
\item This model employs a Specific Temperature Change algorithm \cite{radel_effect_2007, radel_repository_2007} and
\item relies on a supporting \textbf{response database} combining detailed 
spent nuclear fuel composition data \cite{carter_fuel_2011} with a detailed 
thermal repository performance analysis tool from Lawrence Livermore National 
Lab (LLNL) and the Used Fuel Disposition (UFD) 
campaign \cite{greenberg_application_2012}.  
\item This method is capable of rapid estimation of temperature increase near emplacement tunnels as a function of 
\begin{itemize}
\item waste composition,
\item limiting radius, $r_{lim}$, 
\item waste package spacing, $S$, 
\item near field thermal conductivity, $K_{th}$, 
\item and near field thermal diffusivity, $\alpha_{th}$.
\end{itemize}
\end{itemize}

\section{Specific Temperature Change Method}
Introduced by Radel, Wilson et al., the Specific Temperature Change (STC) method uses 
a linear approximation to arrive at the thermal loading density limit 
\cite{radel_repository_2007, radel_effect_2007}.  

First, $\Delta T$ is determined for a limiting loading density 
of the particular material composition then it is normalized to a single 
kilogram of that material, $\Delta t$, the so called STC. 


\section{Specific Temperature Change Superposition}

For an arbitrary waste stream composition, scaled curves, $\Delta t_i$, calculated in this 
manner for individual isotopes can be superimposed for each isotope to arrive at an 
approximate total temperature change.

\begin{align}
 \Delta T (r_{lim}) &\sim \sum_{i} m_i \Delta t_i(r_{lim})
 \label{superposition}
\intertext{where}
 i &= \mbox{ An isotope in the material } [-]\nonumber\\
 m_i &= \mbox{ mass of isotope i  } [kg]\nonumber\\
 \Delta t_i &= \mbox{ Specifc temperature change due to \textsl{i} } [^{\circ}K].\nonumber
\end{align}


\section{Specific Temperature Change Calculations}
\footnotesize{A reference data set of temperature change curves was calculated. 
Repeated runs of a detailed model over the range of values in Table 
\ref{tab:thermal_cases} determined Specific Temperature Change (STC) values 
over that range.

\section{LLNL UFD MathCAD Model}
The analytic model used to populate the reference dataset was created at 
LLNL for the UFD campaign \cite{hardin_generic_2011, 
greenberg_investigations_2012, greenberg_application_2012}. It employs an 
analytic model from Carslaw and Jaeger and is \textbf{implemented in MathCAD}
\cite{carslaw_conduction_1959, ptc_mathcad_2010}.  The integral solver in the 
MathCAD toolset is the primary calculation engine for the analytic MathCAD 
thermal model, which relies on superposition of point, finite-line, and line 
source integral solutions.  


\section{Scaling Demonstration}
As a demonstration of the calculation procedure, the temperature change 
  curve for one initial gram of $^{242}Cm$ and is scaled to represent $25.9g$, 
  approximately the $^{242}Cm$ inventory per MTHM in 51GWd burnup UOX PWR fuel.

\section{Superposition Concept}

The supporting database was limited to some primary heat contributing isotopes 
present in traditional spent nuclear fuel, $H$, 
such that the superposition in equation \eqref{superposition} becomes 


\section{Superposition Demonstration}
As a demonstration of the calculation procedure, scaled temperature change 
  curves for five curium isotopes are superimposed to achieve a total temperature 
change (note log scale).


  \section{Nested Components}
  The NuclideModel in a Component can be interchangeably represented by any of 
  the four nuclide transport models. 

\section{Clay GDSM Model.}

Now that we've addressed the physical layout of the model, I'll add that the 
implemented physics was based on an abstraction process. I had access to a 
detailed computational model of radionuclide transport in a nuclear waste 
repository. This Generic Disposal System Model had the conceptual layout 
expressed by this picture, and involved a full set of engineered barrier 
systems. Though it was very detailed it ran slowly, so the goal of Cyder was 
not to replicate it or simply link to it. Rather, the goal in the Cyder project 
was to capture the dominant physics that are seen in this model. By conducting 
sensitivity analyses with this model, I arrived at important parameters and 
performance responses that informed the model development that follows in this 
section of the talk.

  \section{Advection Dispersion Equation}
  One thing that I found from this abstraction process is that we want to be 
  able to capture mass transfer between the models in advectively dominated, 
  dispersively dominated, or coupled flow regimes.

\section{Timestepping Algorithm}

Each Component passes some information radially outward to the nested 
Component immediately containing it and some information radially 
inward to the nested Component it contains. 

If we begin at timestep $t_n$, the mass distribution and concentration 
profile in an inner component is found according to its mass balance model.

\section{Timestepping Algorithm}

Based on the interface conditions at time $t_n$ in component j and the 
internal conditions in component k at $t_{n-1}$, the mass transfer can be 
found. Here, for some components, the physics of mass transfer can be selected. 
This mass transfer is either advective, dispersive, or coupled. 

\section{Timestepping Algorithm}

Finally, when the appropriate mass is transferred from j to k, components j and 
k both update their mass and concentration profiles accordingly.


  \section{Radionuclide Transport: Degradation Rate Based Release}
In this model, the contaminants in the degraded fraction of the control volume 
are available to adjacent components. The available contaminants
$m_{ij}(t)$, at the boundary between cell $i$ to cell $j$ at time $t$ are thus

  \section{Radionuclide Transport: Degradation Rate Based Release}
<++>

  \section{Radionuclide Transport : Mixed Cell with Sorption and Solubility}
<++>

  \section{Radionuclide Transport : Mixed Cell Sorption}
<++>
The mass of contaminant sorbed into the degraded and precipitated solids can be
found using a linear isotherm model \cite{schwartz_fundamentals_2004},
characterized by the relationship 

  \section{Radionuclide Transport : Mixed Cell Solubility Limitation}
In addition to engineered barriers, contaminant transport is constrained by 
  the solubility limit \cite{hedin_integrated_2002}, 

  \section{Radionuclide Transport: Lumped Parameter Transport Model}
The method by which each lumped parameter component is modeled is
according to a relationship between the incoming concentration, $C_{in}(t)$,
and the outgoing concentration, $C_{out}(t)$.

  \section{Radionuclide Transport: Lumped Parameter Transport Model}
Selection of the response function is usually based on experimental tracer
results in the medium at hand. However, some functions used commonly in
chemical engineering applications \cite{maloszewski_lumped_1996} include the
Piston Flow Model (PFM), Exponental Model (EM), and Dispersion Model (DM). 

  \section{Radionuclide Transport: Lumped Parameter Transport Model}
The solutions to these for constant concentration at the 
source boundary are given by Maolszewski and Zuber.

  \section{Radionuclide Transport: 1D Finite, Cauchy B.C.}
  A one dimensional, finite, unidirectional flow,
  solution with Cauchy and Neumann boundary conditions

  \section{Radionuclide Transport: 1D Finite, Cauchy B.C.}
For the boundary conditions, 

  \section{Radionuclide Transport: 1D Finite, Cauchy B.C.}
For the vertical flow coordinate system, $A$ is defined as

\section{Demonstration}
<++>
\section{Degradation Rate Model Base Case IV}
<++>
\section{Degradation Rate Model Base Case IV}
<++>
\section{Mixed Cell Model Base Case  II}
<++>
\section{Mixed Cell Model Base Case  II}
<++>
\section{Lumped Parameter Model Base Case}
<++>
\section{1dppm Base Case I}
<++>
\section{1dppm Base Case I}
<++>
\section{Clay GDSM Solubility Sensitivity}
For solubility limits above the threshold, increase to the limit had no effect on the peak dose. This demonstrates the 
situation in which the solubility limit is so high that even complete 
dissolution of the waste inventory into the pore water is insufficient to reach 
the solubility limit.
\section{Cyder Solubility Sensitivity}
In the parametric analysis of \Cyder performance, it was shown that the 
solubility sensitivity behavior closely matched that of the GDSM 
sensitivity behaviors. In agreement with expectations, \Cyder results in Figure 
\ref{fig:sol_result}, demonstrate a sharp turnover 
where the solubility limit exceeds the point at which it limits movement. 

\section{Clay GDSM Sorption Sensitivity}
It is clear from Figures \ref{fig:KdSumFactor} and \ref{fig:KdSum} that 
for retardation coefficients greater than the threshold , the 
relationship between peak annual dose and retardation coefficient is a strong 
inverse one. 
\section{Cyder Sorption Sensitivity}

In Figure \ref{fig:kd_result}, increasing the retardation 
coefficient results in a smooth but dramatic turnover. 


\section{Clay GDSM Degradation Rate Sensitivity}

Highly soluble and non-sorbing $^{129}I$ demonstrates a direct proportionality between dose rate and 
fractional degradation rate until a turnover where other natural system 
parameters dampen transport. Highly soluble and non-sorbing $^{129}I$ demonstrates a direct 
proportionality to the inventory multiplier.

$^{129}I$ waste form degradation rate sensitivity.
$^{129}I$ inventory multiplier sensitivity.


\section{Cyder Degradation Rate Sensitivity}
In the parametric sensitivity analysis conducted with the \Cyder tool, waste 
form degradation rate sensitvity similarly shows the two regimes noted in the 
GDSM analysis.  

Sensitivity demonstration of the degradation rate in \Cyder for an 
arbitrary isotope.


\section{GDSM Model Advective Diffusive Sensitivity}
For vertical advective velocities 
$6.31\times10^{-6}[m/yr]$ and above, lower reference diffusivities are 
ineffective at attenuating the mean of the peak doses for soluble, non-sorbing 
elements. 

\section{Cyder Advective Diffusive Sensitivity}
Increased advection and increased diffusion lead to greater release. Also, when 
both are varied, a boundary between diffusive and advective
regimes can be seen. An example of these results are shown here.
Advection vs. Diffusion Sensitivity in Cyder. Dual advective velocity 
and reference diffusivity sensitivity for a non-sorbing, infinitely soluble 
nuclide.

  \section{Thermal Base Case Demonstration}
A validation exercise comparing the combined scaling and  
superposition calculations demonstrates an average error of 1.1\% and a 
maximum error of 4.4\%.

This comparison of STC calculated thermal response from $Cm$ 
inventory per MTHM in 51GWd burnup UOX PWR fuel compares favorably with results 
from the semi-analytic model from LLNL.


  \section{Thermal Base Case Demonstration}
Here, percent error is 
\begin{align}
\mbox{ percent error } &= 100\times\frac{\left|\Delta T_{LLNL} - \Delta 
T_{STC}\right|}{ \Delta T_{LLNL}}.
\end{align}

\section{LLNL Model Waste Package Spacing Sensitivity}
  Figure \ref{fig:Cm242spacing_sens} shows the trend in which increased waste 
  package spacing of a medium decreases areal thermal energy 
  deposition in the near field.
  $K_{th}$ Sensitivity to $s$.
  
  Increased waste package 
  spacing decreases areal thermal energy deposition 
  (here represented by STC) in the near field (here $r_{calc} = 0.5m$).


\section{LLNL Model Limiting Radius Sensitivity}
  The location of the limiting radius has a strong effect on the 
  waste package loading limit, for a fixed limiting temperat
  
  $K_{th}$ Sensitivity to $r_{lim}$.
  
  Increased limiting radius decreases thermal energy deposition contributing to 
  the thermal limit (here represented by STC).

\section{Cyder Spacing and Limiting Radius Sensitivity}
  The thermal diffusivity was compared both with the 
  spacing between waste packages and the limiting radius. 

  $alpha_{th}$ vs. $r_{lim}$ Sensitivity in Cyder.

  Cyder results agree with those of the LLNL model. The importance of the 
  limiting radius decreases with increased $K_{th}$. The above example thermal 
  profile results from 10kg of $^{242}Cm$.

\section{LLNL Model Thermal Conductivity Sensitivity}
By varying the thermal conductivity of the repository model from 0.1 to 4.5 
$[W\cdot m^{-1} \cdot K^{-1}]$, this sensitivity analysis succeeds in capturing 
the domain of thermal conductivities witnessed in high thermal conductivity 
salt deposits as well as low thermal conductivity clays.
$K_{th}$ Sensitivity for Low $\alpha_{th}$ in LLNL Model.
Increased thermal conductivity decreases thermal energy deposition 
(here represented by STC) in the near field (here $r_{calc} = 0.5m$).

\section{Cyder Thermal Conductivity and Limiting Radius Sensitivity}

These figures validate the trend noted above that 
increased thermal conductivity of a medium decreases thermal energy deposition 
in the near field. Additionally, analysis with the \Cyder STC database 
demonstrates the way in which the importance of spacing and the importance of 
the limiting radius decrease with increasing $K_{th}$.

$K_{th}$ vs. $r_{lim}$ Sensitivity in Cyder
Cyder results agree with 
those of the LLNL model. The importance of the limiting radius decreases with 
increased $K_{th}$. The above example thermal profile results from 10kg of 
$^{242}Cm$.

\section{Cyder Thermal Conductivity and Limiting Radius Sensitivity}

$K_{th}$ vs. Waste Package Spacing Sensitivity in Cyder.
Cyder results 
agree with 
those of the LLNL model. The importance of the limiting radius decreases with 
increased $K_{th}$. The above example thermal profile results from 10kg of 
$^{242}Cm$




\section{LLNL Model Thermal Diffusivity Sensitivity}
  By varying the thermal diffusivity of the disposal system from $0.1-3\times 
  10^{-6} [m^2\cdot s^{-1}]$, this sensitivity analysis succeeds in capturing 
  the domain of 
  thermal diffusivities witnessed in high thermal diffusivity salt deposits as 
  well as low thermal diffusivity clays.
    Increased thermal diffusivity decreases thermal energy deposition (here represented by STC) in the near field (here $r_{calc} = 0.5m$).


\section{Cyder Thermal Diffusivity and Conductivity Sensitivity}
$\alpha_{th}$ vs. $K_{th}$ Sensitivity in Cyder. Cyder trends agree
  with those of the LLNL model, in which increased thermal diffusivity results 
  in 
  decreased thermal depsoition in the near field. The above example thermal 
  profile results from 10kg of $^{242}Cm$.

\section{Cyder Thermal Diffusivity and Limiting Radius Sensitivity}
  Further \Cyder analysis shows the importance of $K_{th}$ remains constant, 
  but 
  the importance of the limiting radius decreases with increasing 
  $\alpha_{th}$.
  $alpha_{th}$ vs. $r_{lim}$ Sensitivity in Cyder.
  Cyder trends agree with 
  those of the LLNL model. The importance of the limiting radius decreases with 
  increased $K_{th}$. The above example thermal profile results from 10kg of 
  $^{242}Cm$

\pagebreak
\bibliographystyle{ieeetr}
\bibliography{paper}
\end{document}


