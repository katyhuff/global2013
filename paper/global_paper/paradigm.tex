\section{\textsc{Cyder} Repository Modeling Paradigm}

The \Cyder disposal system simulator architecture is intended to modularly permit 
exchange of disposal system Component models (e.g., detailed nuclide transport 
model vs. less detailed) and data (e.g., exchange clay for granite geologic 
data) and accept arbitrary waste stream isotopic compositions.  
Finally, in order to participate in a \Cyclus simulation as a facility model, \Cyder must 
make requests for spent material up to its capacity. Determination of the 
repository capacity for various types of spent fuel commodities comprises the 
interfacing functionality of the repository model.

\subsection{Waste Stream Acceptance}

The disposal system simulator must accept arbitrary spent fuel and high level waste 
streams. A waste stream is a material data object resulting from the \Cyclus 
simulated fuel cycle.  As radionuclides are gained, lost, and transmuted within 
the spent fuel object, a history of its isotopic composition is recorded.  It 
arrives at the repository and is emplaced if it obeys all repository capacity 
limits. 

For waste streams that vary from each other in composition, the thermal 
capacity of the repository to receive that waste stream must therefore be 
recalculated.  Since disposable material in most simulations of interest will 
be of variable composition and therefore heterogeneous in heat production 
capability, the disposal system simulator will repeatedly need to recalculate its own 
capacity as new materials are offered.  

\subsection{Waste Stream Conditioning}

Waste conditioning is the process of packing a waste stream into an appropriate 
waste form. As \Cyclus lacks a conditioning facility, the \Cyder repository 
fulfills this need as a part of the repository behavior. As a waste stream is 
accepted into the repository, it is associated with a waste form according 
to its commodity name. This pairing is input by the user during simulation 
setup when a number of waste form Component configurations are specified and 
associated with allowed waste stream commodities. It is according to these 
pairings that \Cyder loads discrete waste forms with discrete waste 
stream contaminant vectors as depicted in Figure \ref{fig:ws_conditioning}.

\begin{figure}[htbp!]
\begin{center}
\def\svgwidth{.5\columnwidth}
\input{./paradigm/ws_conditioning.eps_tex}
\end{center}
\caption[Waste stream conditioning in \Cyder.]{Waste streams are accepted and 
conditioned into the appropriate waste form according to user-specified 
pairings between commodities and waste forms.}
\label{fig:ws_conditioning}
\end{figure}

\subsection{Waste Form Packaging}

Waste packaging is the process of placing one or many waste forms into a 
containment package (typcially metallic). Once the waste stream has been 
conditioned into a waste form, that waste form Component is loaded into a waste 
package Component, also according to allowed pairs dictated by the user, as 
depicted in Figure \ref{fig:wf_packaging}.

\begin{figure}[htbp!]
\begin{center}
\def\svgwidth{.5\columnwidth}
\input{./paradigm/wf_packaging.eps_tex}
\end{center}
\caption[Waste packaging in \Cyder.]{Waste forms are loaded into the 
appropriate waste package according to user-specified pairings between forms 
and packages.}
\label{fig:wf_packaging}
\end{figure}


\subsection{Package Emplacement}

Finally, the waste package is emplaced in a buffer component, which 
contains many other waste packages, spaced evenly in a grid. The grid is 
defined by the user input and depends on repository depth, $\Delta z$, waste 
package spacing, $\Delta x$, and tunnel spacing, $\Delta y$ as in Figure 
\ref{fig:repo_layout}.

\begin{figure}[htbp!]
\begin{center}
\def\svgwidth{.5\columnwidth}
\input{./paradigm/repo_layout.eps_tex}
\end{center}
\caption[The gridded \Cyder repository emplacement geometry.]{The \Cyder 
repository emplacement geometry allows generic representation of all semi-regular 
two dimensional gridded layouts.}
\label{fig:repo_layout}
\end{figure}

\subsection{Nested Components}

The fundamental unit of information in the disposal system simulator is radionuclide 
contaminant presence at each stage of containment.  The disposal system simulator, in 
this way, is fundamentally a tool to determine thermal and contaminant 
transport evolution as a result of an arbitrary waste stream. The disposal 
system simulator in this work conducts this calculation by  treating each containment 
Component as a nested volume in a release chain. 

Each Component is defined by a Geometry, some Material Data, a ThermalModel, 
and a NuclideModel. It is also defined by the Parent Component which contains 
it and the daughter components which it contains. An emplaced waste package 
Component, for example, possesses a pointer to the buffer that surrounds it, 
its Parent Component. It also possesses a list of pointers to the waste form or 
waste forms within it, its daughter components. 

\subsubsection{Component Geometry}

Each Component of the repository system (i.e. waste form, waste package, buffer, 
and geologic medium) is modeled as a discrete control volume. Each control 
volume performs its own mass balance at each time step and assesses its own 
internal  heat transfer and degradation phenomena utilizing boundary condition 
information provided by adjacent nested Components. This control volume is 
defined by the Component Geomtry, a class which keeps track of the inner and 
outer radii, length, and centroid coordinates of the (assumed cylindrical) 
volume.

\subsubsection{Component Material Data}

Each Component of the repository system possesses a notion of the material that 
it is made of. Supporting thermal and hydrologic data for canonical engineered 
barrier and geologic media is provided with the code in the 
mat\_data.sqlite SQLite database.

Each table in the database holds data related to one of a canonical set of 
engineered barrier and geologic medium materials (e.g. clay, glass, etc.).  
The columns of that table hold data required to support all \Cyder models. 
Thermal diffusivity and thermal conductivity comprise the thermal data in the 
table for each material. The hydrologic and chemical data in the database has 
one table for each material. Each table contains relative diffusivity 
coefficients, solubility limits, and sorption parameters for each element.  

\subsubsection{Component ThermalModel}

Each Component possesses a thermal transport model that determines the 
temperature inside the Component over time. This allows limitations within any 
barrier at any limiting radius to control the limiting thermal response within 
the \Cyder repository simulator. 

\subsubsection{Component NuclideModel}

Each Component possesses a radionuclide contaminant transport model that 
determines the contaminant transport inside the Component over time. The 
choices available for this NuclideModel range in fidelity but emphasize the 
distinction between dominant transport modes (advective or disperive) in 
various geologic media as well as dominant retarding geochemistry for specific 
isotopes (sorption and solubility limitation).

\subsubsection{Implicit Timestepping}

Each Component passes some information radially outward to the nested 
Component immediately containing it and some information radially 
inward to the nested Component it contains. 


In the case of radionuclide transport, for example, each Component model
requires information about the radionuclides released from the Component it
immediately contains.  Thus, nuclide release information is passed radially
outward from the waste stream sequentially through each containment layer to
the geosphere. However, the solutions within each Component often rely on the
external boundary conditions of that Component.  Thus, the \Cyder model uses an
implicit timestepping method to arrive at the future state of each Component,
radially outward, as a function of both the past state and the current state. 

That is, in Component j, some Component in a nested series, the mass flux 
entering the Component at time $t_n$ is found from the initial state of the cell 
at time $t_n$, the inner boundary 
condition at time $t_n$ and the outer boundary condition at $t_{n-1}$.  

\begin{align}
  \dot{m}_{ij}^n &= f( m_j(t_{n-1}) , BC_i(t_n) , BC_j(t_{n-1}) . . . ) \nonumber\\
  \intertext{where}
  m_{ij}(t_n) &= \mbox{ contaminant mass flux from component i to j }[kg/timestep]\nonumber\\
  BC_i(t_n)  &= \mbox{ inner conditions at }r_i\mbox{, and time }t_n \nonumber \\
  BC_j(t_{n-1})  &= \mbox{ outer conditions at }r_j\mbox{, and time }t_{n-1} \nonumber\\
  f &= \mbox{ functional form of contaminant transport into j. }\nonumber
\end{align}

Once the mass flux into the component is found, the mass is removed from the 
inner cell, updating its state in preparation for the next time step.

\begin{align}
  m_i^\dagger(t_n)  &= m_i(t_n)  - m_{ij}(t_n) 
  \intertext{where}
  m_i^\dagger(t_n)  &= \mbox{ updated mass in component i }[kg]
\end{align}

In this way, the contained mass in the component is described as
\begin{align}
  m_j(t_n)  &= m_j(t_{n-1})  + \dot{m}_j(t_n) . \nonumber
\end{align}

Resulting concentration profiles across the component can then be calculated 
and one can solve, numerically, for the outer boundary condition at $t_n$ 

\begin{align}
  BC_j(t_n) &= g\left( m_j(t_n) , C_j(t_n) \right)\nonumber\\
  g &= \mbox{functional form of contaminant transport across j}\nonumber
\end{align}

This boundary condition can, in turn, be used by the component external to it, $k$ as the $t_n$ 
inner boundary condition of its own solution and so on.

