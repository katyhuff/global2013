
\section{Introduction}

This work demonstrates the sensitivity of fuel cycle system metrics to dynamic 
feedback effects of geologic disposal capacity constraints and repository performance. Cyder 
\cite{huff_cyder_2013} is a computational tool capable of dynamically 
determining repository capacity and performance for arbitrary waste compositions 
in arbitrarily comingled waste streams in a range of reducing, saturated 
geologic disposal environments (i.e.  enclosed clay, salt, and granite 
concepts). In this work the Cyder tool integrates dynamically with the  Cyclus 
fuel cycle simulation tool \cite{wilson_cyclus:_2012} to demonstrate the 
magnitude with which global fuel cycle system metrics are sensitive to thermal 
and hydrologic repository geology parameters in the context of fuel cycle 
technology decisions.

Sensitivity analyses were conducted which span the parametric range described in Table 
\ref{tab:base_cases}.

\begin{table}[ht!]
\centering
\footnotesize{
\begin{tabular}{|l|l|l|r|}
\multicolumn{4}{c}{\textbf{Sensitivity Analysis Cases}}\\
\hline
\textbf{Parameter} & \textbf{Symbol} & \textbf{Units} & \textbf{Value Range} \\
\hline
Thermal & & & \\
Diffusivity & $\alpha_{th}$ & $[m^2\cdot s^{-1}]$ & $1.0\times10^{-7}-3.0\times10^{-6}$\\
\hline
Thermal & & & \\
Conductivity & $K_{th}$     & $[W\cdot m^{-1} \cdot K^{-1}]$ & $0.1 - 4.5$ \\
\hline
Package & & & \\
Spacing & $S$ & $[m]$ & 2-50 \\
\hline
Thermal & & & \\
Limiting & & & \\
Radius & $r_{lim}$ & $[m]$ & 0.1-5 \\
\hline
Hydraulic & & & \\
Reference & & & \\
Diffusivity& $\alpha_{h,ref}$& $[m^2s]$ & $10^{8} - 10^{15}$ \\
\hline
Hydraulic & & & \\
Conductivity& $K_{h}$& $[m \cdot s^{-1}]$ & $10^{-13} - 10^{-3}$ \\
\hline
Advective  & & & \\
Water & & & \\
Velocity & $v_{adv}$ & $[m\cdot s^{-1}]$ & $2\times10^{-16}-2\times10^{-12}$ \\
\hline
Porosity & $\theta$ & $[\%]$ & 0.1 - 60\\
\hline
Effective & & & \\
Porosity & $\theta_{eff}$ & $[\%]$ & $5\times10^{-6}-5$\\
\hline
Sorption \& & & & Reducing - \\
Behavior & $K_{d,i}$& $[m^3\cdot kg^{-1}]$ & Oxidizing \\
\hline
Solubility &  & & Reducing -\\
Limitation & $C_{sol,i}$ & $[kg\cdot m^{-3}3]$& Oxidizing \\
\hline
\end{tabular}
\caption{The sensitivity analyses conducted in this work covered a range of 
thermal and hydrologic geology parameters in the context of canonical fuel cycle choices.}
\label{tab:base_cases}
}
\end{table}



Fuel cycle metrics sensitive to repository capacity feedbacks include average spent fuel storage time, necessary 
intermediate storage capacity, and availability of reusable spent fuel. Other 
global fuel cycle behaviors that are affected by repository feedbacks include 
material routing and spent fuel market economics. This sensitivity analysis 
focused on average spent fuel storage time and intermediate storage capacity as 
well as necessary repository footprint and peak annual dose, which are affected 
by heat and radionuclide release characteristics specific to variable spent fuel 
compositions associated with alternative fuel cycles.  


\section{Methods}


For dynamic thermal capacity analysis in Cyder, a transient model utilizes a 
linear approximation of heat based capacity quickly for each 
arbitrary waste stream offered to the repository. This relies on a thermal reference database of repository heat 
evolution curves covering the thermal coefficient range of the main geologies of 
interest and over a range of realistic waste package spacings. 

Hydrologic contaminant transport in Cyder is implemented with four 
interchangeable  methods in a modular software design. These modeling options 
alternately optimize speed and fidelity in representations of barrier components 
within the repository concept (i.e. waste form, waste package, buffer, near 
field geology, and far field geology)\cite{huff_hydrologic_2013}.  Simplistic 
models include a congruent release component degradation model and a mixed cell 
control volume model. For systems in which the flow can be assumed constant, a 
medium fidelity lumped parameter dispersion model is implemented. Also 
implemented is a Leij et al. solution to the advection dispersion equation for 
Cauchy boundary condition \cite{leij_analytical_1991, 
van_genuchten_analytical_1982}.  

\section{Results}

Computational fuel cycle simulation tools capable of simulating heterogeneous spent fuel 
isotopics resulting from transition scenarios and alternative fuel cycles have 
previously lacked equally dynamic repository modeling options.  This work 
demonstrates that the Cyder generic repository model can inform fuel cycle 
metrics by providing just such a dynamic interface between waste management and fuel cycle technology decisions. 

The sensitivity of global fuel cycle metrics concerned with material routing, 
intermediate storage time, and repository performance were investigated in the 
context of dynamic feedback effects from repository capacity and performance. 
