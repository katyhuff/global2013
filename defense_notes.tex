%        File: defense_notes.tex
%     Created: Tue Jul 30 03:00 PM 2013 C
% Last Change: Tue Jul 30 03:00 PM 2013 C
%
\documentclass[letterpaper]{article}
\usepackage[top=1.0in,bottom=1.0in,left=1.0in,right=1.0in]{geometry}
\usepackage{verbatim}
\usepackage{amssymb}
\usepackage{graphicx}
\usepackage{longtable}
\usepackage{amsfonts}
\usepackage{amsmath}
\usepackage[usenames]{color}
\usepackage[
naturalnames = true, 
colorlinks = true, 
linkcolor = Black,
anchorcolor = Black,
citecolor = Black,
menucolor = Black,
urlcolor = Blue
]{hyperref}
\def\thesection       {\arabic{section}}
\def\thesubsection     {\thesection.\alph{subsection}}

\author{K. Huff
\\ \href{mailto:khuff@cae.wisc.edu}{\texttt{khuff@cae.wisc.edu}}
}

\date{}
\title{Defense Notes}
\begin{document}
\maketitle

\textbf{Title}


\textbf{Outline}


\section{Motivation}

\textbf{Top Level Fuel Cycle Simulators}


\subsection{Future Fuel Cycle Options}
\textbf{Future Fuel Cycle Options}



\subsection{Geologic Disposal Options}
\textbf{Disposal Geology Options Considered}



\textbf{Repository Components}



\subsection{Fuel Cycle Simulator Capabilities}
\textbf{Need for an Integrated Repository Model}


\textbf{Outline}

\section{Modeling Paradigm}
\textbf{Cyder Paradigm: Modularity}
A modular repository framework facilitates
\begin{itemize}
\item interchangable subcomponents
\item and simulations with varying levels of detail.
\end{itemize}
Integration with a fuel cycle simulator facilitates
\begin{itemize}
\item analysis of feedback effects upon the fuel cycle
\item and fuel cycle impacts on disposal system performance.
\end{itemize}

\section{Demonstration}



\section*{LLNL Model Waste Package Spacing Sensitivity}
  Figure \ref{fig:Cm242spacing_sens} shows the trend in which increased waste 
  package spacing of a medium decreases areal thermal energy 
  deposition in the near field.
  $K_{th}$ Sensitivity to $s$.
  
  Increased waste package 
  spacing decreases areal thermal energy deposition 
  (here represented by STC) in the near field (here $r_{calc} = 0.5m$).


\section*{LLNL Model Limiting Radius Sensitivity}
  The location of the limiting radius has a strong effect on the 
  waste package loading limit, for a fixed limiting temperat
  
  $K_{th}$ Sensitivity to $r_{lim}$.
  
  Increased limiting radius decreases thermal energy deposition contributing to 
  the thermal limit (here represented by STC).

\section*{Cyder Spacing and Limiting Radius Sensitivity}
  The thermal diffusivity was compared both with the 
  spacing between waste packages and the limiting radius. 

  $alpha_{th}$ vs. $r_{lim}$ Sensitivity in Cyder.

  Cyder results agree with those of the LLNL model. The importance of the 
  limiting radius decreases with increased $K_{th}$. The above example thermal 
  profile results from 10kg of $^{242}Cm$.

\section*{LLNL Model Thermal Conductivity Sensitivity}
By varying the thermal conductivity of the repository model from 0.1 to 4.5 
$[W\cdot m^{-1} \cdot K^{-1}]$, this sensitivity analysis succeeds in capturing 
the domain of thermal conductivities witnessed in high thermal conductivity 
salt deposits as well as low thermal conductivity clays.
$K_{th}$ Sensitivity for Low $\alpha_{th}$ in LLNL Model.
Increased thermal conductivity decreases thermal energy deposition 
(here represented by STC) in the near field (here $r_{calc} = 0.5m$).

\section*{Cyder Thermal Conductivity and Limiting Radius Sensitivity}

These figures validate the trend noted above that 
increased thermal conductivity of a medium decreases thermal energy deposition 
in the near field. Additionally, analysis with the \Cyder STC database 
demonstrates the way in which the importance of spacing and the importance of 
the limiting radius decrease with increasing $K_{th}$.

$K_{th}$ vs. $r_{lim}$ Sensitivity in Cyder
Cyder results agree with 
those of the LLNL model. The importance of the limiting radius decreases with 
increased $K_{th}$. The above example thermal profile results from 10kg of 
$^{242}Cm$.

\section*{Cyder Thermal Conductivity and Limiting Radius Sensitivity}

$K_{th}$ vs. Waste Package Spacing Sensitivity in Cyder.
Cyder results 
agree with 
those of the LLNL model. The importance of the limiting radius decreases with 
increased $K_{th}$. The above example thermal profile results from 10kg of 
$^{242}Cm$




\section*{LLNL Model Thermal Diffusivity Sensitivity}
  By varying the thermal diffusivity of the disposal system from $0.1-3\times 
  10^{-6} [m^2\cdot s^{-1}]$, this sensitivity analysis succeeds in capturing 
  the domain of 
  thermal diffusivities witnessed in high thermal diffusivity salt deposits as 
  well as low thermal diffusivity clays.
    Increased thermal diffusivity decreases thermal energy deposition (here represented by STC) in the near field (here $r_{calc} = 0.5m$).


\section*{Cyder Thermal Diffusivity and Conductivity Sensitivity}
$\alpha_{th}$ vs. $K_{th}$ Sensitivity in Cyder. Cyder trends agree
  with those of the LLNL model, in which increased thermal diffusivity results 
  in 
  decreased thermal depsoition in the near field. The above example thermal 
  profile results from 10kg of $^{242}Cm$.

\section*{Cyder Thermal Diffusivity and Limiting Radius Sensitivity}
  Further \Cyder analysis shows the importance of $K_{th}$ remains constant, 
  but 
  the importance of the limiting radius decreases with increasing 
  $\alpha_{th}$.
  $alpha_{th}$ vs. $r_{lim}$ Sensitivity in Cyder.
  Cyder trends agree with 
  those of the LLNL model. The importance of the limiting radius decreases with 
  increased $K_{th}$. The above example thermal profile results from 10kg of 
  $^{242}Cm$


\section{Conclusion}
\textbf{<++>}
\textbf{<++>}
\textbf{<++>}

An integrated repository performance model enables the analysis of global fuel
cycle metrics concerned with material routing, intermediate storage time, and
repository performance. Post-processed analysis of these metrics can neglect
feedback effects of geologic disposal capacity constraints and repository
performance within the fuel cycle system.


\pagebreak
\bibliographystyle{ieeetr}
\bibliography{paper}
\end{document}


