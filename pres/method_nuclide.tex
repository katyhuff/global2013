
\begin{frame}
\frametitle{Clay GDSM Sensitivity Analysis}
\footnotesize{
\begin{itemize}
\item Barrier Degradation
\item Sorption
\item Solubility
\item Advective Velocity
\item Diffusivity
\end{itemize}
\input{./images/clay_gdsm}
}
\end{frame}

\begin{frame}
  \frametitle{Nested Components}
  The NuclideModel in a Component can be interchangeably represented by any of 
  the four nuclide transport models. 
    \begin{itemize}
      \item Degradation Rate Based Failure Model
      \item Mixed Cell with Degradation, Sorption, Solubility Limitation
      \item Lumped Parameter Model
      \item 1 Dimensional Approximate Advection Dispersion Solution, Brenner 
      \cite{brenner_diffusion_1962}
    \end{itemize}
\end{frame}



%\begin{frame}
%  \frametitle{Component Interfaces}
%  \footnotesize{
%Solutions to this equation can be categorized by their boundary conditions and 
%those boundary conditions serve as the interfaces between components in the 
%Cyder library of nuclide transport models.
%
%  \begin{figure}[htp!]
%    \begin{center}
%      \def\svgwidth{\textwidth}
%      \input{images/flow.eps_tex}
%    \end{center}
%    \caption{The boundaries between components are robust interfaces defined by 
%    Source Term, Dirichlet, Neumann, and Cauchy boundary conditions.}
%    \label{fig:flow}
%  \end{figure}
%  }
%\end{frame}

\begin{frame}
  \frametitle{Radionuclide Transport: Degradation Rate Based Release}
  \input{./images/deg_volumes}
\end{frame}



\begin{frame}
  \frametitle{Radionuclide Transport : Mixed Cell with Sorption and Solubility}
  \input{./images/deg_sorb_volumes}
\end{frame}

\begin{frame}
  \frametitle{Radionuclide Transport : Mixed Cell Sorption}
The mass of contaminant sorbed into the degraded and precipitated solids can be
found using a linear isotherm model \cite{schwartz_fundamentals_2004},
characterized by the relationship 
\begin{align}
s_{i} &= K_{di} C_{i}
\label{linear_iso}
\intertext{where}
s_i &= \mbox{ the solid concentration of isotope i }[kg/kg]\nonumber\\
K_{di} &= \mbox{ the distribution coefficient of isotope i}[m^3/kg]\nonumber\\
C_i &= \mbox{ the liquid concentration of isotope i }[kg/m^3].\nonumber
\end{align}
\end{frame}


\begin{frame}
  \frametitle{Radionuclide Transport : Mixed Cell Solubility Limitation}
In addition to engineered barriers, contaminant transport is constrained by 
  the solubility limit \cite{hedin_integrated_2002}, 
    \begin{align}
      m_{s,i} &\leq V_w C_{sol,i},
    \intertext{where}
      m_{s,i} &= \mbox{ solubility limited mass of isotope i in volume }V_w [kg]\nonumber\\ 
      V_w &= \mbox{ volume of the solution }[m^3]\nonumber\\
      C_{sol,i} &= \mbox{ solubility limit, the maximum concentration of i }[kg/m^3].\nonumber
    \end{align}
\end{frame}


\begin{frame}
  \frametitle{Radionuclide Transport: Lumped Parameter Transport Model}
\footnotesize{
\begin{figure}[htbp!]
  \begin{center}
    \def\svgwidth{\textwidth}
    \input{images/lumpedseries.eps_tex}
  \end{center}
  \caption{ The method by which each lumped parameter component is modeled is
according to a relationship between the incoming concentration, $C_{in}(t)$,
and the outgoing concentration, $C_{out}(t)$.}
  \label{fig:lumpedseries}
\end{figure}

\begin{align}
  C_{out}(t) &= \int_0^\infty C_{in}(t-t')g(t')e^{-\lambda t'}dt'
  \label{lumped2}
  \intertext{where}
  t'  &= \mbox{ time of entry }[s]\nonumber\\
  t-t'  &= \mbox{ transit time }[s]\nonumber\\
  g(t-t')  &= \mbox{ response function, a.k.a. transit time distribution}[-]\nonumber\\
  \lambda &= \mbox{ radioactive decay constant}[s^{-1}].\nonumber
\end{align}
}
\end{frame}

\begin{frame}
  \frametitle{Radionuclide Transport: 1D Finite, Cauchy B.C.}
\begin{figure}[htbp!]
  \begin{center}
    \def\svgwidth{.5\textwidth}
    \input{images/1dfin.eps_tex}
  \end{center}
  \caption{A one dimensional, finite, unidirectional flow,
  solution with Cauchy and Neumann boundary conditions 
\cite{van_genuchten_analytical_1982, brenner_diffusion_1962}.}
  \label{fig:1dinf}
\end{figure}
\end{frame}


\input{./nuclide_demonstration/wf_deg_inv}
\input{./nuclide_demonstration/kd}
\input{./nuclide_demonstration/sol}
