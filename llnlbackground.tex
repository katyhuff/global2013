% LLNL
\subsection{Analytical Model Background}
\begin{frame}[ctb!]
\frametitle{Analytical Model : Background}
The analytical  model
\begin{itemize} 
  \item was created at LLNL (H. Greenberg, J. Blink, et. al) \cite{hardin_generic_2011, greenberg_investigations_2012, 
greenberg_application_2012}
  \item employs an analytic model from Carslaw and Jaeger \cite{carslaw_conduction_1959} 
  \item is implemented in MathCAD \cite{ptc_mathcad_2010}
  \item seeks to inform heat limited waste capacity calculations for 
    \begin{itemize}
      \item arbitrary geology 
      \item arbitrary waste package loading densities
      \item arbitrary homogeneous decay heat source
    \end{itemize}
\end{itemize}
\end{frame}

\begin{frame}
  \frametitle{Analytical Model : Geometry}
  \begin{figure}[h!]
    \begin{center}
      \includegraphics[width=0.7\textwidth]{./images/llnlConcept.eps}
    \caption{Vertical, horizontal, alcove, and borehole emplacement layouts can 
    be represented by a line of point sources and adjacent line sources 
    \cite{greenberg_investigations_2012}.}
    \label{fig:llnl}
    \end{center}
  \end{figure}
\end{frame}

\begin{frame}
  \frametitle{Analytical Model : Calculation Method}
    LLNL's model is a MathCAD solution of the transient homogeneous 
    conduction equation,
    \begin{align}
      \nabla^2T  = \frac{1}{\alpha}\frac{\partial T}{\partial t},
      \label{condGl}
    \end{align}
    in which superimposed point and line source solutions approximate the repository 
    layout.
\end{frame}

\begin{frame}[ctb!]
\frametitle{Analytical Model : Calculation Method}
The model consists of two conceptual regions, an external region representing 
the host rock and an internal region representing the waste form, package, and 
buffer Engineered Barrier System within the disposal tunnel wall.   
\begin{itemize}
  \item Since the thermal mass of the EBS is small in comparison to the thermal 
    mass of the host rock, the internal region may be treated as quasi-steady 
    state.
  \item The transient state of the temperature at the calculation radius is 
    found with a convolution of the transient external solution with the steady 
    state internal solution.
  \item The internal and external regions are \textbf{approximated} to be a 
    single homogeneous medium.
  \item The process is then iterated with a one year resolution in order to 
    arrive at a temperature evolution over the lifetime of the repository. 
\end{itemize}
\end{frame}


\begin{frame}[ctb!]
\frametitle{Analytical Model : Calculation Method}
\begin{minipage}{0.3\textwidth}
\begin{figure}[h!]
  \begin{center}
    \includegraphics[width=\textwidth]{./images/llnlConcept.eps}
  \end{center}
  \caption{The central package is represented by a finite line source
  \cite{greenberg_investigations_2012}.}
  \label{fig:llnl}
\end{figure}
\end{minipage}
\hspace{0.01mm}
\begin{minipage}{0.6\textwidth}
The geometric layout of the analytic LLNL model in Figure \ref{fig:llnl} 
shows  that the central package is represented by the finite line solution
\footnotesize{
\begin{align}
  T_{line}&(t,x,y,z) = \nonumber\\
  &\frac{1}{8\pi K_{th}} 
  \bigintsss_0^t\!\frac{q_L(t')}{t-t'}e^{ \frac{-\left(x^2 + z^2\right)}{4\alpha 
  (t-t')} }\nonumber\\ &\cdot\left[ \erf{\left[ \frac{1}{2} \frac{\left( y + 
  \frac{L}{2} \right)}{\sqrt{\alpha(t-t')}}  \right]} - \erf{\left[ \frac{1}{2} 
  \frac{\left( y - \frac{L}{2} \right)}{\sqrt{\alpha(t-t')}}  \right]} 
  \right]\,\mathrm{dt'}.
  \label{line}
\end{align}
}
\end{minipage}
\end{frame}

\begin{frame}[ctb!]
\frametitle{Analytical Model : Calculation Method}
\begin{minipage}{0.3\textwidth}
\begin{figure}[h!]
  \begin{center}
    \includegraphics[width=\textwidth]{./images/llnlConcept.eps}
  \end{center}
  \caption{Adjacent packages are represented as point sources
  \cite{greenberg_investigations_2012}.}
  \label{fig:llnl}
\end{figure}
\end{minipage}
\hspace{0.1mm}
\begin{minipage}{0.6\textwidth}
 Adjacent packages within the central tunnel are represented by the point source 
 solution,
 \footnotesize{
  \begin{align}
    T_{point}(t,r) &= \frac{1}{8K_{th}\sqrt{\alpha}\pi^{\frac{3}{2}}}\nonumber\\
     &\bigintsss_0^{t}\!\frac{q(t')}{(t-t')^{\frac{3}{2}}}e^{\frac{-r^2}{4\alpha(t-t')}}\,\mathrm{dt'}.
    \label{point}
  \end{align}
  }
  \end{minipage}
\end{frame}


\begin{frame}[ctb!]
\frametitle{Analytical Model : Calculation Method}
\begin{minipage}{0.3\textwidth}
\begin{figure}[h!]
  \begin{center}
    \includegraphics[width=\textwidth]{./images/llnlConcept.eps}
  \end{center}
  \caption{The non-central disposal tunnels are represented as infinite line sources
  \cite{greenberg_investigations_2012}.}
  \label{fig:llnl}
\end{figure}
\end{minipage}
\hspace{0.1mm}
\begin{minipage}{0.6\textwidth}
Adjacent disposal tunnels are represented by the infinite line source solution,
\footnotesize{
\begin{align}
  T_{\infty line}(t,x,z) &= \frac{1}{4\pi K_{th}} 
  \bigintsss_0^t\frac{q_L(t')}{t-t'}e^{ \frac{-\left(x^2 + z^2\right)}{4\alpha 
  (t-t')} }
  \label{infline}
  \intertext{in infinite homogeneous media, where}
  \alpha &= ~~\mbox{thermal diffusivity } [m^2\cdot s^{-1}]\nonumber\\
  q(t) &= ~~\mbox{point heat source} [W]\nonumber\\
  \intertext{and}
  q_L(t) &= ~~\mbox{linear heat source} [W\cdot m^{-1}]\nonumber
\end{align}
}
Superimposed point and line source solutions allow for a notion of the 
repository layout to be modeled in the host rock.
\end{minipage}
\end{frame}


